\documentclass[12pts]{book}
\usepackage[utf8]{inputenc}
\usepackage[T1]{fontenc}
\usepackage[french]{babel}

%Mise en page du document
\usepackage[top=1.5cm,bottom=1.5cm,left=1.5cm,right=1cm]{geometry}


%Ce qui concerne l'auteur
\author{HELOU Komlan Mawulé}
\date{\today}
\title{PREPA BAC: MATHEMATIQUES}


%Les packages importés
\usepackage{amsmath,wasysym,amssymb}
\usepackage{tabularx}
\usepackage{tikz}


%Début du document: il sera éditer en deux partie. La première les exercices et la deuxièmes les corrigés
\begin{document}
	\maketitle
	%Première partie: elle comprend 6 chapitres(Nombres complexes et transformations du plan, suite numérique, statistique, Dénombrement et probabilité, Equation différentielle, Programmation linéaire et l'étude de fonction nyumérique)%
	\part{RECCUEILS EXERCICES}
	\chapter{Nombres complexes et transformations du plan}
	%Début de l'exercice 1
	\section*{EXERCICE 1}
		On considère dans l'ensemble $\mathbb{C}$ des nombres complexes, le polynôme $P$ définie par: $P(z)=z^3+(-7+2i)z^2+(15-4i)z-25+10i$.
		\begin{enumerate}
			\item 
				\begin{enumerate}
					\item Vérifier que: $P(5-2i)=0$
					\item Résoudre dans $\mathbb{C}$ l'équation $P(z)=0$.
				\end{enumerate}
			\item \begin{enumerate}
					Soit $S$ la similitude plane directe de centre $I$ d'affixe $z_1=-3-2i$ et qui transforme le point $A$ d'affixe $z_A=1+2i$ en $B$ d'affixe $z_B=5-2i$.
					\begin{enumerate}
						\item Déterminer $f$, l'application complexe associée à $S$
						\item Déterminer les éléments caractéristiques de $S$.
					\end{enumerate}
				\end{enumerate}
		\end{enumerate}
	%Fin de l'exercice 1
	
	%Début de l'exercice 2
	\section*{EXERCICE 2}
		On considère l'équation : $(E): z^3-9z^2+(22+12i)z-12-36i=0$
		\begin{enumerate}
			\item Démontrer que l'équation $(E)$ admet une solution réelle $z_1$ et une solution imaginaire pure $z_2$.
			\item Résoudre dans $\mathbb{C}$ l'équation $(E)$
			\item Dans le plan complexe muni d'un repère orthonormé $(o,\overrightarrow{u},\overrightarrow{v})$, on considère les points $A$, $B$ et $C$ d'affixes respectives $z_A=3$, $z_B=2i$ et $z_C=6-2i$.
			\begin{enumerate}
				\item Placer les points $A$, $B$ et $C$ dan le plan complexe.
				\item Montrer que: $\frac{z_C-z_A}{z_B-z_A}$ est réel. Que peut-on en déduire?
				\item Soit $S$ la similitude directe plane d'angle $\frac{\pi}{4}$ et de rapport $\sqrt{2}$ transformant A en B.\\
				Donner l'écriture complexe de $S$ et préciser son centre $\Omega$
			\end{enumerate}
		\end{enumerate}
	%Fin de exercice 2
	
	%Début de l'exercice 3
	\section*{EXERCICE 3}
		Le plan complexe est muni d'un repère orthonormé direct $(O,\overrightarrow{u},\overrightarrow{v})$. On considère l'application $f$ définie sur $\mathbb{C}^*$ par: $f(z)=\frac{1}{3}(z+\frac{1}{Z})$.
		\begin{enumerate}
			\item On désigne par $K$ le point d'affixe $f(\frac{1}{2}+i\frac{\sqrt{3}}{2})$. Déterminer les coordonnées de K.
			\item Soit $\alpha$ un nombre réel. Résoudre dans $\mathbb{C}$ l'équation $(E): f(z)=\frac{2}{3}\cos \alpha$.
			\item 
				\begin{enumerate}
					\item En déduire les solutions dans $\mathbb{C}$ de l'équation $(E'): z^4-2(\cos \alpha)z^2+1=0$ (On donnera les solutions sous forme exponentielle)
					\item Vérifier que les solutions de $(E')$ sont deux à deux conjugués.
					\item Décomposer le polynôme à variable réelle $x$ définie par: $P(x)=x^4-2(\cos \alpha )x^2+1$ en  un produit de deux polynômes de second degré à coefficients réels.
				\end{enumerate} 
			\item	On considère l'application$h$ du plan complexe dans lui-même qui a tout point $M$ d'affixe $z$ associe le point $M'$ d'affixe telle $z'$ telle que: $2(z-\frac{1}{3})=(1+i)(z'-\frac{1}{3})$.
				\begin{enumerate}
					\item  Démontrer que $h$ est une similitude plane directe dont on précisera les éléments caractéristiques.
					\item Démontrer que $h$ est une composée d'une rotation et d'une homothétie dont on donnera les éléments caractéristiques.
				\end{enumerate}
		
		\end{enumerate}
	%Fin de l'exercice 3
	
	%Début de l'exercice 4
	\section*{EXERCICE 4}
		\begin{enumerate}
			\item 
				\begin{enumerate}
					\item Ecrire sous forme algébrique le nombre complexe $(1+i\sqrt{3})^2$.
					\item Résoudre dans $\mathbb{C}$ l'équation: $z^2+2z+19-18i\sqrt{3}=0$
				\end{enumerate}
			\item Résoudre dans $\mathbb{C} \times \mathbb{C}$ le système suivant:
				$\left\{ \begin{array}{l}
					z_1 + z_2 =-2 \\
					z_1^2-z_2^2=-4+i\frac{4\sqrt{3}}{3}
				\end{array} \right.$
				\item Dans le plan complexe rapporté au repère orthonormal $(O,\overrightarrow{u}, \overrightarrow{v})$, on désigne par $A$, $B$, $C$ et $D$ les points d'affixes respectives $a$, $b$, $c$ et $d$ telle que $a=2+3i\sqrt{3}$; $b=\frac{1}{9}(\overline{a} - 2)$; $c=-a-2$  et $d= a-b-c$ (où $\overline{a}$ est le conjugué de a).
			\begin{enumerate}
				\item Déterminer les nombres complexes $b$, $c$ et $d$ puis placer les points $A$, $B$, $C$ et $D$ dans le plan complexe. (Unité graphique: 2cm et $\sqrt{3}\cong 1,7$)
				\item Comparer $a+b$ et $b+d$. En déduire la nature du quadrilatère $ABCD$.
				\item Calculer et interpréter géométriquement un argument du nombre complexe $Z=\frac{c-a}{d-b}$. Préciser la nature exacte de $ABCD$.
			\end{enumerate}
		\end{enumerate}
	%Fin de l'exercice 4
	
	%Début de l'exercice 5
	\section*{EXERCICE 5}
	\begin{enumerate}
		\item Dans le plan complexe muni d'un repère orthonormé $(O,\overrightarrow{u},\overrightarrow{v})$ on considère les points $A$, $B$, $C$ et $D$ d'affixes respectives: $z_A=-i$, $z_B=1+i$, $z_C=-1+2i$ et $z_D=-2$.
			\begin{enumerate}
				\item Placer sur une figure les points $A$, $B$, $C$ et $D$.
				\item \begin{enumerate}
						\item Interpréter géométriquement le module et l'argument du nombre complexe $z=\frac{z_A-z_C}{z_B-z_D}$.
						\item Calculer le nombre complexe $z$.
				   	  \end{enumerate}
				\item Déterminer le module et l'argument de $z$ puis en déduire la nature du quadrilatère $ABCD$
			\end{enumerate}
		\item Soit $\lambda$ un nombre complexe de module $1$ différent de $1$. On définit, pour tout entier naturel $n$ la suite $(z_n)$ de nombres complexes par: $\left\{\begin{array}{l}
																		z_0=0\\
																		z_{n+1}=\lambda z_n-i
																	\end{array}   \right.$
		On note $M_n$ le point d'affixe $z_n$.
			\begin{enumerate}
				\item \begin{enumerate}
						\item Calculer $z_1$, $z_2$ et $z_3$
						\item Démontrer que $\forall n \in \mathbb{N}^*$,\quad $z_n=-(\lambda^{n-1}+\lambda^{n-2}+ \dots +\lambda+1)i$
						\item En déduire que $\forall n \in \mathbb{N}^*, \quad z_n=\frac{1-\lambda^n}{\lambda-1}i$
					\end{enumerate}
				\item On suppose que pour tout entier naturel $k$ tel que $\lambda^k=1$. Démontrer que pour tout entier naturel, $z_{n+k}=z_k$.
				\item Etude du cas $\lambda=i$
					\begin{enumerate}
						\item Montrer que $\forall n \in \mathbb{N}, \forall k \in \mathbb{N}, z_k=z_{n+k}$.
						\item Montrer que $M_{n+1}$ est l'image de $M_n$ par la rotation $\varphi$ dont on précisera le centre et l'angle.
						\item Déterminer les images de $A$, $B$, $C$ et $D$ par $\varphi$ et placer dans le repère précédent ces images.
						\item Montrer que $\forall n\in \mathbb{N}, z_{4n+1}=i$
					\end{enumerate}
			\end{enumerate}
	\end{enumerate}
	%Fin de l'exercice 5
	%Fin du premier chapitre
	
	%début du cahpitre 2
	\chapter{Dénombrement et probabilité}
	\section*{EXERCICE 1}	
	Une société "Gnamienlait" de Gnamien produit des sachets de lait caillé. Soit X la variable aléatoire qui associe chaque sachet de lait caillé produit, sa masses en gramme (g). La loi de probabilité de X est définie par le tableau ci-dessous:
	\begin{tabular}{|c|c|c|c|c|c|c|c|}
		\hline 
		x(g) & 220 &230 & 240 & 250 & 260 & 270 & 280 \\ \hline 
		$p_i$ & 0,08& 0,10 & $a$ &$b$& 0,16 & 0,15 & 0,04 
	\\	\hline
	\end{tabular}
	$a$ et $b$ sont deux nombres réels.
	$w_i$ représente la masse du sachet de lait caillé; $p_i$ la probabilité qu'un sachet de lait ait la masse$x$.
	\begin{enumerate}
		\item \begin{enumerate}
			\item Calculer $\mathbb{E}(X)$ l'espérance mathématique de $X$ en fonction de $a$ et $b$
			\item Sachant que $\mathbb{E}(X)=250$, justifie que $a=0,14$ et $b=0,33.$ 
		\end{enumerate}
	Dans la suite de l'exercice, on considère les valeurs de $a$ et $b$ données ci-dessus.
	\item Gnamien prend au hasard un sachet de lait caillé de sa société.  Calculer la probabilité pour que la masse de ce sachet de lait caillé soit au moins 250\\
	Tiéplé, la fille de Gnamien, prend au hasard de façon indépendante cins sachets de lait caillé.\\
	Calculer la probabilité qu'elle ait choisi exactement trois sachets de lait caillé de 220g.\\
	On prendra l'arrondi d'ordre 3 du résultat.
	\item Cette machine est réglée pour éliminer en principe les sachets de lait de masse inférieur à 250g.
		\begin{enumerate}
			\item[$\bullet$] Si un sachet de lait caillé à 240g, la probabilité qu'il soit éliminé est de 0,7.
			\item[$\bullet$] Si un sachet de lait caillé à 230g, la probabilité qu'il soit éliminé est de 0,8.
			\item[$\bullet$] Si un sachet de lait caillé à 220g, il est systématiquement éliminé.
			\item[$\bullet$] Si un sachet  de lait caillé à une masse supérieur ou égale à 250g, il est systématiquement accepté.
		\end{enumerate}
	\begin{enumerate}
		\item Justifier que la probabilité qu'un sachet de lait de 240g soit éliminé est de 0,098.
		\item Calculer la probabilité pour qu'un sachet de lait caillé de cette société soit éliminé.
	\end{enumerate}
	\end{enumerate}
	%Fin de l'exercice 1
	
	%Début de l'exercice 2
	\section*{EXERCICE 2}
	On teste un médicament sur un ensemble d'individus ayant un taux de glycémie anormalement élevé. Pour cela, $60 \%$ des individus prennent le médicament, les autres recevant une substance neutre et l'on étudie à l'aide d'un test la baisse du taux de glycémie.\\
	Chez les individus ayant pris le médicament, on constate une baisse de taux avec une probabilité de 0,8. On ne constate aucune baisse de ce taux pour $90\%$  des personnes ayant reçu la substance neutre.
	\begin{enumerate}
		\item Calculer la probabilité d'avoir une baisse de taux de glycémie sachant qu'on a pris le médicament.
		\item Démontrer que la probabilité d'avoir une baisse de taux de glycémie est 0,52.
		\item On soumet au test un individu pris au hasard. Quelle est la probabilité qu'il ait pris le médicament sachant que l'on constate une baisse de taux de glycémie?
		\item On contrôle 5 individus au hasard.
		\begin{enumerate}
			\item Quelle est la probabilité d'avoir exactement deux personnes dont le taux de glycémie a baissé?
			\item Quelle est la probabilité d'avoir au moins un individu dont le taux de glycémie a baissé?
		\end{enumerate}
		\item On contrôle $n$ individus pris au hasard. ($n$ est un nombre entier naturel non nul)\\
		Déterminer $n$ pour que la probabilité d'avoir au moins un individus dont le taux de glycémie a baissé soit supérieur à 0,98.
	\end{enumerate}
	%Fin de l'exercice 2
	
	%Début de l'exercice 3
	\section*{EXERCICE 3}
	Une urne contient une boule rouge, deux boules blanches et trois boules noires indiscernables au toucher.\\
	On tire au hasard successivement et avec remise trois boules de l'urne.\\
	Soit les événement suivantes:\\
	A: "les trois boules tirées sont de même couleur"\\
	B: "Il n'y a aucune boule blanche parmi les boules tirées.\\
	C: "Il y a exactement deux boules blanches parmi les boules tirées."
	\begin{enumerate}
		\item Monter que $p(A)=\frac{1}{6}$ et $p(B)$$\frac{8}{27}$
		\item Calculer $p(C)$
	\end{enumerate}
	%Fin de l'exercice 3
	
	%Début de l'exercice 4
	\section*{EXERCICE 4}
	Une urne contient huit boules indiscernables au toucher: cinq boules blanches, trois boules rouges et deux boules vertes (Voir la figure ci-dessous)\\
	\begin{center}
		\begin{tikzpicture}
			\draw[thick] (-4,4)--(-4,0);
			\draw[thick] (-4,0)--(4,0);
			\draw[thick] (4,0)--(4,4);
			\draw[fill=green] (-3,1) circle [radius = 0.5];
			\draw[fill=green] (-3,3) circle [radius = 0.5];
			\draw[fill=red] (-1.75,1) circle [radius = 0.5];
			\draw[fill=red] (-1.75,3) circle [radius = 0.5];
			\draw[fill=red] (-1,2) circle [radius = 0.5];
			\draw[fill=white] (0,1) circle [radius = 0.5];
			\draw[fill=white] (0,3) circle [radius = 0.5];
			\draw[fill=white] (1.5,1) circle [radius = 0.5];
			\draw[fill=white] (1.5,3) circle [radius = 0.5];
			\draw[fill=white] (3,2) circle [radius = 0.5];
		\end{tikzpicture}
	\end{center}
	On tire au hasard et simultanément quatre boules de l'urne.
	\begin{enumerate}
		\item Soit A l'événement "Parmi les quatre boules tirées il y a une seule boule verte" et B l'événement "Parmi les quatre boules tirées, il y a exactement trois boules de même couleur". Montrons que $p(A)=\frac{8}{15}$ et $p(B)=\frac{19}{70}$
		\item Soit $X$ la variable aléatoire qui à chaque tirage associe le nombre de boules vertes tirées.
		\begin{enumerate}
			\item Montrer que $p(X=2)=\frac{2}{15}$
			\item Déterminer la loi de probabilité de la variable $X$ et montrer que l'espérance mathématique est égale à $\frac{4}{5}.$
		\end{enumerate}
	\end{enumerate}
%Fin de l'exercice 4

%Début de l'exercice 5
\section*{EXERCICE 5}
	Une étude statistique montre que dans une ville donnée; $15\%$ des individus âgés de moins de 60 ans et $80\%$ des individus âgés de plus 60 ans ont été vaccinés contre la grippe.\\ Les individus âgés de plus de 60 ans représentent $30\%$ de la population. On choisit au hasard, une personne de cette population et on considère les événements suivants:\\
	\begin{enumerate}
		\item[$\bullet$] G: \fg la personne est âgés de plus de 60 ans\og{}.
		\item[$\bullet$] V: \fg La personne est vaccinée\og{}.
	\end{enumerate}
	\begin{enumerate}
		\item Recopier et compléter l'arbre de probabilité ci-dessous:\\
		\begin{center}
			\begin{tikzpicture}
				\draw[thick, ->] (-4,0) -- (-2,1) node[anchor=west] {G};  
				\draw[thick, ->] (-4,0) -- (-2,-1) node[anchor=west] {$\overline{G}$};
				\draw[thick, ->] (-1.5,1) -- (1,1.75) node[anchor=west] {V};
				\draw[thick, ->] (-1.5,1) -- (1, 0.25) node[anchor=west] {$\overline{V}$};
				\draw[thick, ->] (-1.5,-1) -- (1,-0.25) node[anchor=west] {V};
				\draw[thick, ->] (-1.5,-1) -- (1, -1.75) node[anchor=west] {$\overline{V}$};
			\end{tikzpicture}
		\end{center}
	\item Montrer que la probabilité pour qu'une personne soit vacciné est égale à 0,345.
	\item La personne choisie étant vaccinée, quelle est la probabilité pour qu'elle soit âgée de moins de 60 ans.
	\item On choisit au hasard 10 personnes âgées de plus de 60ans. Calculer la probabilité pour que deux exactement d'entre elles soient vaccinées.
	\item On choisit, au hasard, $n$ personnes âgées de plus de 60 ans.
		\begin{enumerate}
			\item Quelle est la probabilité pour qu'aucune d'entre elles ne soit vaccinée?
			\item Déterminer la probabilité $p_n$ pour que l'une au moins d'entre elle soit vaccinée.
			\item Déterminer la plus petite valeur de $n$ pour que $p_n \geq 0,9$
		\end{enumerate}
	\end{enumerate}

\end{document}